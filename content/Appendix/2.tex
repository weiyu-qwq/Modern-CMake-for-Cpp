这个指令提供了关于列表的基本操作:读取、搜索、修改和排序。某些模式会改变列表(改变原始值)。若以后需要它,请确保复制原始值。

完整的细节可以在在线文档中找到:

\url{https://cmake.org/cmake/help/latest/command/list.html}

\hspace*{\fill} \\ %插入空行
\noindent
\textbf{读取}

有以下几种模式:

\begin{itemize}
\item 
list(LENGTH <list> <out>) 计算<list>变量中的元素,并将结果存储在<out>变量中。

\item 
list(GET <list> <index>... <out>) 将<list>元素与<index>索引列表一起复制到<out>变量。

\item 
list(JOIN <list> <glue> <out>) 用<glue>分隔符<list>元素,并将结果字符串存储在<out>变量中。

\item 
list(SUBLIST <list> <begin> <length> <out>) 与GET模式类似,但操作范围而不是显式索引。若<length>为-1,则返回<list>变量中提供的从<begin>索引到列表末尾的元素。
\end{itemize}

\hspace*{\fill} \\ %插入空行
\noindent
\textbf{搜索}

该模式只是在<list>变量中查找<needle>元素的索引,并将结果存储在<out>变量中(若没有找到该元素,则为-1):

\begin{lstlisting}[style=styleCMake]
list(FIND <list> <needle> <out>)
\end{lstlisting}

\hspace*{\fill} \\ %插入空行
\noindent
\textbf{修改}

有以下几种模式:

\begin{itemize}
\item 
list(APPEND <list> <element>...) 将一个或多个<element>值添加到<list>变量的末尾。

\item 
list(PREPEND <list> [<element>...]) 与APPEND类似,但在<list>变量的开头添加元素。

\item 
list(FILTER <list> {INCLUDE | EXCLUDE} REGEX <pattern>) 过滤<list>变量以包含或排除值匹配<pattern>的元素。

\item 
list(INSERT <list> <index> [<element>...]) 在给定的<index>处向<list>变量添加一个或多个<element>值。

\item 
list(POP\_BACK <list> [<out>...]) 从<list>变量末尾删除一个元素,并将其存储在可选的<out>变量中。若提供了多个<out>变量,则将删除更多元素来填充它们。

\item 
list(POP\_FRONT <list> [<out>...]) 类似于POP\_BACK,但从<list>变量的开头删除一个元素。

\item 
list(REMOVE\_ITEM <list> <value>...) FILTER EXCLUDE的简写,但不支持正则表达式。

\item 
list(REMOVE\_AT <list> <index>...) 从<list>中移除特定<index>处的元素。

\item 
list(REMOVE\_DUPLICATES <list>) 从<list>中删除重复项。

\item 
list(TRANSFORM <list> <action> [<selector>] [OUTPUT\_VARIABLE <out>]) 对<list>元素应用特定的转换。

默认情况下,该动作应用于所有元素,但可以通过添加<selector>来限制效果。若提供了OUTPUT\_VARIABLE关键字,提供的列表将发生变化(就地更改),结果将存储在<out>变量中。

以下选择器可用:AT <index>,FOR <start> <stop>[<step>],以及REGEX <pattern>。操作包括APPEND <string>, PREPEND <string>,TOLOWER,TOUPPER,STRIP,GENEX\_STRIP和REPLACE  <expression>。工作原理与具有相同名称的string()模式完全相同。
\end{itemize}


\hspace*{\fill} \\ %插入空行
\noindent
\textbf{排序}

有以下几种模式:

\begin{itemize}
\item 
list(REVERSE <list>) 简单地颠倒<list>的顺序。

\item 
list(SORT <list>) 按字典序对列表进行排序。更多高级选项请参考在线手册。
\end{itemize}

