CMake还支持一些简单的算术操作。详细信息请参见在线文档:

\url{https://cmake.org/cmake/help/latest/command/math.html}

要计算一个数学表达式并将其作为可选<格式>(十六进制或十进制)的字符串存储在<out>变量中,请使用以下签名:

\begin{lstlisting}[style=styleCMake]
math(EXPR <out> "<expression>" [OUTPUT_FORMAT <format>])
\end{lstlisting}

<expression>值是一个字符串,它支持C代码中出现的操作符(它们在这里有相同的含义):

\begin{itemize}
\item 
算术: +, -, *, /, \% 模除运算

\item 
位操作: | or, \& and, \^{} xor, ~ not, <{}< 左移位, >{}> 右移位

\item 
括号 (...)
\end{itemize}

常量值可以用十进制或十六进制格式提供。













