
可以在CMake中构建的最基本目标是包含整个应用程序的单个二进制可执行文件,其可由一段源码组成,例如经典的helloworld.cpp。也可以是一些复杂的东西——由数百甚至数万个文件构建。这就是许多初级项目看起来的样子——用一个源文件创建一个二进制文件,添加另一个,然后,所有内容都链接到单个二进制文件,没有任何文件结构。

作为软件开发人员,我们故意划定边界并指定组件来分组一个或多个翻译单元(.cpp文件)。这样做的原因有很多:增加代码可读性,管理耦合和连接,加快构建过程,最后提取可重用组件。每个足够大的项目都需要引入某种形式的分区。

CMake中的目标就是这个问题的解决方案——为CMake形成单一目标的高级逻辑单元。一个目标可能依赖于其他目标,并且可以以声明的方式产生的。CMake将负责确定构建目标的顺序,然后逐个执行必要的步骤。作为一般规则,构建一个目标将产生一个工件,该工件将发送到其他目标或作为构建的最终产品交付。

故意使用“工件”这个不准确的词,是因为CMake不限制只生成可执行文件或库。实际上,可以使用生成的构建系统来创建多种类型的输出:更多的源文件、头文件、目标文件、存档和配置文件——可以是任何东西。我们所需要的只是一个命令行工具(如编译器)、可选的输入文件和一个输出路径。

目标是一个非常强大的概念,极大地简化了项目的构建。理解其如何工作,以及如何以最优雅和干净的方式配置它们非常重要。

本章中,我们将讨论以下主题:

\begin{itemize}
\item 
目标的概念

\item 
编写自定义命令

\item 
生成器表达式
\end{itemize}