有读者可能认为,在我们成功地将源代码编译为二进制文件之后,作为构建工程师的工作就完成了。几乎就是这样——二进制文件包含CPU要执行的所有代码,但代码以非常复杂的方式分散在多个文件中。链接是一个简化事情的过程,使机器代码简洁、快速地使用。

快速浏览一下命令列表,就会发现CMake并没有提供那么多与链接相关的命令。不可否认,target\_link\_libraries()是唯一真正配置此步骤的程序。那么,为什么要用一整章的篇幅来阐述一条命令呢?不幸的是,在计算机科学中几乎没有什么是容易的,链接也不例外。

为了获得正确的结果,需要遵循规则——理解链接器的工作原理并掌握正确的基本知识。我们将讨论目标文件的内部结构,重定位和引用解析如何工作,以及它的用途。我们将讨论最终可执行文件与它的组件之间的区别,以及系统如何构建流程映像。

然后,我们将向您介绍各种库—静态、动态和模块,它们都称为库,但它们几乎没有什么相似之处。构建正确链接的可执行文件很大程度上依赖于有效的配置(并注意位置无关代码(PIC)等小细节)。

我们将了解链接的另一个麻烦——单一定义规则(ODR),需要让定义的数量完全正确。处理重复的符号有时是非常棘手的,特别是在使用动态库时。然后,将了解为什么链接器有时无法找到外部符号,即使在可执行文件与适当的库链接时也是如此。

最后,将了解如何节省时间,并使用链接器来准备使用专用框架进行测试的解决方案。

本章中,我们将讨论以下主题:

\begin{itemize}
\item 
掌握正确的链接方式

\item 
构建不同类型的库

\item 
用定义规则解决问题

\item 
连接顺序和未定义符号

\item 
分离main()进行测试
\end{itemize}