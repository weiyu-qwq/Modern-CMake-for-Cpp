解决方案是大是小并不重要。随着它的成熟,将最终决定引入外部依赖项。避免使用主流业务逻辑创建和维护代码的成本是很重要的。这样,就可以把时间花在对你和客户都重要的事情上。

外部依赖不仅用于提供框架和特性,还用于解决古怪的问题。还可以在构建和控制代码质量的过程中发挥重要作用——无论是以特殊编译器(如Protobuf)的形式,还是以测试框架(如GTest)的形式。

无论使用的是开源项目,还是使用公司中其他开发人员编写的项目,仍然需要一个良好的、干净的过程来管理外部依赖关系。自己解决这个问题将需要无数小时的设置和大量额外的支持工作。幸运的是,CMake在适应依赖管理的不同风格和历史方法方面做得很好,同时跟上行业批准的标准的不断发展。

要提供一个外部依赖项,应该首先检查主机系统是否已经有这个依赖项可用,因为最好避免不必要的下载和冗长的编译。我们将探索如何找到并将这些依赖项转换为CMake目标,以便在我们的项目中使用。这可以通过多种方式实现,特别是当包支持开箱即用CMake或至少为稍老的PkgConfig工具提供文件时。不过,我们仍然可以编写自己的文件来检测并包含这样的依赖项。

我们将讨论当系统上没有依赖项时该怎么办?可以采取其他步骤来自动提供必要的文件。我们将考虑使用不同的Git方法来解决这个问题,并将整个CMake项目作为构建的一部分。

本章中,我们将讨论以下主题:

\begin{itemize}
\item 
如何查找已安装的软件包

\item 
使用FindPkgConfig0发现遗留包

\item 
编写自己的查找模块

\item 
使用Git库

\item 
使用ExternalProject和FetchContent模块
\end{itemize}
