
编译可以大致描述为将用高级编程语言编写的指令,转换为低级机器代码的过程。可以使用抽象概念(如类和对象)创建应用程序,而不必费心处理特定于处理器的汇编语言的细节。从而不需要直接使用CPU寄存器,不需要考虑长短跳,也不需要管理堆栈帧。编译语言表达能力更强、可读性更强、更安全,并培养更易维护的代码(但仍然尽可能保持性能)。

C++依赖于静态编译——整个程序在执行之前必须翻译成本机代码。这是Java或Python等语言的另一种替代方法,后者在用户每次运行程序时,都会使用一个特殊的、单独的解释器动态编译程序。每种方法都有自己的优点。C++的策略是提供尽可能多的高级工具,同时仍然能够在一个完整的、自包含的应用程序中,为所有的架构提供原生性能。

创建和运行C++程序需要几步:

\begin{enumerate}
\item 
设计应用程序,并仔细编写源码。

\item 
将单个的.cpp实现文件(称为翻译单元)编译为目标文件。

\item 
将目标文件链接到一个可执行文件中,并添加所有其他依赖项——动态库和静态库。

\item 
为了运行该程序,操作系统将使用一个名为加载器的工具,将其机器码和所有必需的动态库映射到虚拟内存。然后加载器读取头文件以检查程序从哪里开始,并将控制权移交给代码。

\item 
C++运行时启动。执行一个special\_start函数来收集命令行参数和环境变量。启动线程,初始化静态符号,并注册清理回调。这样,才能调用main()(其中代码由开发者书写)。
\end{enumerate}

相当多的工作发生在幕后,本章主要关注于列表中的第二步。通过全面考虑,可以更好地理解一些可能的问题来自哪里。毕竟,软件中并没有什么黑魔法(即使难以逾越的复杂性让它看起来是这样)。每件事都有一个解释和理由。程序运行时期间,由于编译的方式,可能会失败(即使编译步骤本身已经成功通过)。编译器不可能在工作过程中检查所有的边界情况。

\subsubsubsection{5.2.1\hspace{0.2cm}编译工作}

编译是将高级语言转换为低级语言的过程——是通过以特定于给定平台的二进制目标文件格式生成机器代码(特定处理器可以直接执行的指令)。在Linux上,最流行的格式是可执行和可链接格式(ELF)。Windows使用PE/COFF格式规范。macOS使用的是Mach对象(Mach-O格式)。

目标文件是单个源文件的直接翻译。其中的每一个都必须单独编译,然后通过链接器连接到一个可执行程序或库中。所以,当修改代码时,可以通过只重新编译刚修改的文件来节省时间。

编译器必须执行以下步骤来创建一个目标文件:

\begin{enumerate}
\item 
预处理

\item 
语言分析

\item 
汇编

\item 
优化

\item 
生成二进制文件
\end{enumerate}

预处理(由大多数编译器自动调用)是实际编译前的一步。作用是以非常基本的方式操作源代码,可执行\#include指令,用定义的值替换标识符(\#define指令和-D标志),使用简单的宏,并有条件地包含或排除基于\#if、\#elif和\#endif指令的部分代码。预处理器并不了解C++代码,所以其只是一个高级点的查找和替换工具。尽管如此,其工作在构建高级的程序方面还是至关重要的;将代码分解成多个部分,并跨多个翻译单元共享声明的能力,是代码重用的基础。

接下来是语言分析。编译器将逐个字符扫描文件(包含预处理器包含的所有头文件)并执行词法分析,将其分组为有意义的标记——关键字、操作符、变量名等。然后,将记号分组到记号链中,并验证它们的顺序和存在是否遵循C++的规则——这个过程称为语法分析或解析(通常,这是打印错误方面最引人注目的部分)。最后,执行语义分析——编译器试图检测文件中的语句是否真正有意义。例如,必须满足类型正确性检查(不能将整数赋值给字符串变量)。

汇编是根据平台可用的指令集,将这些记号转换为特定于CPU的指令。有些编译器实际上创建一个汇编输出文件,该文件稍后可以传递给特定的汇编程序,以生成CPU可以执行的机器代码。其他的则直接从内存生成相同的机器代码。通常,这样的编译器包含一个选项,可以生成人类可读的汇编文本(这样做可能不值得)。

优化发生在整个编译过程中,在每个阶段都有。生成第一版汇编后有一个显式的阶段,该阶段负责最小化寄存器的使用并删除不使用的代码。一个有趣且重要的优化是内联扩展或内联。编译器将“剪切”函数体并“粘贴”它,而不是调用(标准没有定义在什么情况下会发生这种情况——这取决于编译器的实现)。这个过程加快了执行速度并减少了内存的使用,但不利于调试(执行的代码不再位于原来的行)。

生成二进制文件由根据目标平台指定的格式将优化的机器代码写入目标文件组成。这个目标文件还没有准备好执行——必须传递给下一个工具,链接器,其将适当地重新定位我们的目标文件的部分,并解析对外部符号的引用。这是从ASCII源代码转换为处理器可识别的二进制目标文件。

这些阶段中的每一个都是重要的,可以配置以满足我们的特定需求。看看如何使用CMake管理这个过程。

\subsubsubsection{5.2.2\hspace{0.2cm}初始配置}

CMake提供了多个指令来影响每个阶段:

\begin{itemize}
\item 
target\_compile\_features(): 需要具有特定功能的编译器来编译此目标。

\item 
target\_sources(): 向已定义的目标添加源。

\item 
target\_include\_directories(): 设置预处理器的包含路径。

\item 
target\_compile\_definitions(): 设置预处理定义。

\item 
target\_compile\_options(): 特定于编译器的选项。

\item 
target\_precompile\_headers(): 预编译头文件。
\end{itemize}

以上所有指令都接受类似的参数:

\begin{lstlisting}[style=styleCMake]
target_...(<target name> <INTERFACE|PUBLIC|PRIVATE>
	<value>)
\end{lstlisting}

所以,其支持属性传播,并且可以用于可执行程序和库。另外,还要提醒一下——所有指令都支持生成器表达式。

\hspace*{\fill} \\ %插入空行
\noindent
\textbf{要求编译器提供特定的特性}

正如第3章中所讨论的那样,需要为出错做好准备,并致力于为软件的用户提供明确的信息——可用的编译器X并没有提供所需的特性y。这比用户在不兼容的工具链中所产生的错误要好得多。我们不希望用户认为这是代码的问题,而不是他们的环境出了问题。

下面的指令允许指定目标构建所需的所有特性:

\begin{lstlisting}[style=styleCMake]
target_compile_features(<target> <PRIVATE|PUBLIC|INTERFACE>
						<feature> [...])
\end{lstlisting}

CMake很了解C++标准,并支持这些compiler\_ids的编译器特性:

\begin{itemize}
\item 
AppleClang: Apple Clang for Xcode versions 4.4+

\item 
Clang: Clang Compiler versions 2.9+

\item 
GNU: GNU Compiler versions 4.4+

\item 
MSVC: Microsoft Visual Studio versions 2010+

\item 
SunPro: Oracle Solaris Studio versions 12.4+

\item 
Intel: Intel Compiler versions 12.1+
\end{itemize}

\begin{tcolorbox}[colback=blue!5!white,colframe=blue!75!black,title=重要的Note]
当然,可以使用CMAKE\_CXX\_KNOWN\_FEATURES变量,但建议使用通用C++标准——cxx\_std\_98, cxx\_std\_11, cxx\_std\_14, cxx\_std\_17, cxx\_std\_20,或cxx\_std\_23。查看扩展阅读部分了解更多细节。
\end{tcolorbox}

\subsubsubsection{5.2.3\hspace{0.2cm}管理目标源}

已经知道如何告诉CMake哪些源文件组成了单个目标——可执行文件或库。当使用add\_executable()或add\_library()时,需要提供文件列表。

随着解决方案的增长,每个目标的文件列表也会增长。最后可能会得到一些非常冗长的add\_…()指令。应该如何应对呢?一个诱惑可能是在GLOB模式下使用file()指令——可以从子目录收集所有文件并将它们存储在一个变量中。可以将它作为参数传递给目标声明,而不必再费心于列表文件:

\begin{lstlisting}[style=styleCMake]
file(GLOB helloworld_SRC "*.h" "*.cpp")
add_executable(helloworld ${helloworld_SRC})
\end{lstlisting}

但不建议这种方法。CMake根据列表文件中的更改生成构建系统,所以若不做更改,构建可能会在没有任何警告的情况下中断(从长时间的调试经验中我们知道,这是最糟糕的中断类型)。除此之外,没有在目标声明中列出所有的源代码将破坏诸如CLion (CLion只解析一些命令来理解项目)等IDE的代码检查。

若不建议在目标声明中使用变量,如何有条件地添加源文件,例如:当处理特定平台的实现文件,如gui\_linux.cpp和gui\_windows.cpp?

可以使用target\_sources()指令追加源文件到之前创建的目标:

\begin{lstlisting}[style=styleCMake]
# chapter05/01-sources/CMakeLists.txt

add_executable(main main.cpp)
if(CMAKE_SYSTEM_NAME STREQUAL "Linux")
	target_sources(main PRIVATE gui_linux.cpp)
elseif(CMAKE_SYSTEM_NAME STREQUAL "Windows")
	target_sources(main PRIVATE gui_windows.cpp)
endif()
\end{lstlisting}

这样,每个平台都可以获得自己的一组兼容文件,但是长长的文件列表应该如何处理呢?好吧,我们只能接受有些东西还不完美,并继续手动添加它们。

既然已经了解了关于编译的关键知识点,接下来让我们更深入地了解第一步——预处理。























