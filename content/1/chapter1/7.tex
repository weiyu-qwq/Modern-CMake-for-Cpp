Now you understand what CMake is and how it works; you learned the key components of the CMake tool family and how to install them on a variety of systems. Like a true power user, you know all the ways in which to run CMake through the command line: buildsystem generation, building a project, installing, running scripts, command-line tools, and printing help. You are aware of the CTest, CPack, and GUI applications. This will help you to create projects, with the right perspective, for users and other developers. Additionally, you learned what makes up a project: directories, listfiles, configs, presets, and helper files, along with what to ignore in your VCS. Finally, you took a sneak peek at other non-project files: standalone scripts and modules.

In the next chapter, we will take a deep dive into CMake's programming language. This will allow you to write your own listfiles and open the door to your first script, project, and module.