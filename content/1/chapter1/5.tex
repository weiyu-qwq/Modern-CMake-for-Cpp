
CMake使用相当多的文件来管理它的项目。修改文件内容之前,让我们尝试对每个文件的作用有一个大致的了解。重要的是,即使文件包含CMake语言命令,也不能确定它是为开发人员编辑而设计的。生成一些文件供后续工具使用,对这些文件所做的修改都将在某个阶段重写。高级用户根据自己的需要,使用其他文件对项目进行调整。最后,还有一些临时文件在特定的上下文中可以提供有价值的信息。本节还将说明它们中的哪些不应该位于版本控制系统中。

\subsubsubsection{1.5.1\hspace{0.2cm}源码树}

这是项目所在的目录(也称为项目根目录),包含了所有的C++源码和CMake项目文件。

以下是这个目录的几个关键点:

\begin{itemize}
\item 
需要在其顶部目录中提供一个CMakeLists.txt配置文件。

\item 
使用VCS(例如git)来管理它。

\item 
该目录的路径可有使用cmake命令的-S参数给定。

\item 
不要在CMake代码中硬编码源树的绝对路径——软件的用户可以将项目存储在不同的路径下。
\end{itemize}

\subsubsubsection{1.5.2\hspace{0.2cm}构建树}

CMake使用这个目录来存储构建过程中生成的所有内容:项目的构件、瞬态配置、缓存、构建日志,以及本地构建工具将创建的文件。此目录的可选名称包括构建树。

以下是这个目录的几个关键点:

\begin{itemize}
\item 
二进制文件将在这里创建,例如可执行文件和库,以及用于最终链接的目标文件和存档。

\item 
不要将这个目录添加到VCS——其特定于系统。

\item 
CMake推荐源外构建或在与所有源文件分离的目录中生成工件的构建。通过这种方式,可以避免使用临时的、系统特定的文件(或源内构建)污染项目的源码树。

\item 
可由-B指定,若提供了到源码的路径,则作为cmake命令的最后一个参数,例如,\texttt{cmake -S ../project ./.}

\item 
建议项目包含安装阶段,可以将最终工件放在系统中正确的位置,这样所有用于构建的临时文件都可以删除。
\end{itemize}


\subsubsubsection{1.5.3\hspace{0.2cm}listfiles}

包含CMake语言的文件称为listfiles,可以通过使用include()和find\_package(),或间接使用add\_subdirectory()来将其包含在另一个文件中:

\begin{itemize}
\item 
CMake不会强制对这些文件进行一致的命名,通常的扩展名为.cmake。

\item 
主CMakeLists.txt的文件非常重要,它是在配置阶段执行的第一个文件,需要放在源树的顶层。

\item 
当CMake遍历源码树,并包含不同的listfiles时,内置了以下变量:CMAKE\_CURRENT\_LIST\_DIR、CMAKE\_CURRENT\_LIST\_FILE、CMAKE\_PARENT\_LIST\_FILE和CMAKE\_CURRENT\_LIST\_LINE。
\end{itemize}

\subsubsubsection{1.5.4\hspace{0.2cm}CMakeLists.txt}

CMake项目配置为CMakeLists.txt列表文件,需要在源码树根部提供一个。这样的顶层文件是在配置阶段首先执行的,它应该至少包含两个命令:

\begin{itemize}
\item 
cmake\_minimum\_required(VERSION <x.xx>): 设置CMake的期望版本(并隐式地告诉CMake要对遗留行为应用哪些策略)。

\item 
project(<name> <OPTIONS>): 这用于命名项目(提供的名称将存储在PROJECT\_NAME中)并指定配置它的选项(将在第2章中进一步讨论这个问题)。
\end{itemize}

随着软件的发展,可能希望将其划分为更小的单元,以便分别进行配置和推理。CMake通过子目录和其自身的CMakeLists.txt文件来完成划分,项目结构可能如下所示:

\begin{tcblisting}{commandshell={}}
CMakeLists.txt
api/CMakeLists.txt
api/api.h
api/api.cpp
\end{tcblisting}

可以使用一个非常简单的CMakeLists.txt文件将它们组合在一起:

\begin{lstlisting}[style=styleCMake]
cmake_minimum_required(VERSION 3.20)
project(app)
message("Top level CMakeLists.txt")
add_subdirectory(api)
\end{lstlisting}

项目的主要方面包含在顶层文件中:管理依赖项、声明需求和检测环境。这个文件中,还有一个add\_subdirectory(api)指令来包含另一个CMakeListst。执行特定于应用api部分的指令。

\subsubsubsection{1.5.5\hspace{0.2cm}CMakeCache.txt}

当配置阶段第一次运行时,缓存变量将从列表文件中生成并存储在CMakeCache.txt中。该文件位于构建树的根目录中:

\begin{lstlisting}[style=styleCMake]
# This is the CMakeCache file.
# For build in directory:
c:/Users/rapha/Desktop/CMake/empty_project/build
# It was generated by CMake: C:/Program
Files/CMake/bin/cmake.exe
# You can edit this file to change values found and used by
cmake.
# If you do want to change a value, simply edit, save, and
exit the editor.
# The syntax for the file is as follows:
# KEY:TYPE=VALUE
# KEY is the name of a variable in the cache.
# TYPE is a hint to GUIs for the type of VALUE, DO NOT EDIT
TYPE!.
# VALUE is the current value for the KEY.
########################
# EXTERNAL cache entries
########################
//Flags used by the CXX compiler during DEBUG builds.
CMAKE_CXX_FLAGS_DEBUG:STRING=/MDd /Zi /Ob0 /Od /RTC1
// ... more variables here ...
########################
# INTERNAL cache entries
########################
//Minor version of cmake used to create the current loaded
cache
CMAKE_CACHE_MINOR_VERSION:INTERNAL=19
// ... more variables here ...
\end{lstlisting}

从标题中的注释中观察到的,这种格式不言自明。EXTERNAL部分中的缓存项是供用户修改的,而INTERNAL部分是由CMake管理的。注意,不建议手动更改它们。

以下是需要记住的几个要点:

\begin{itemize}
\item 
可以通过调用cmake(请参阅掌握命令行部分中的缓存选项)或通过ccmake/cmakegui管理该文件。

\item 
可以通过删除该文件将项目重置为默认配置,将从listfiles中重新生成。

\item 
缓存变量可以从列表文件中读写。有时候,变量参考的评估有点复杂,将在第2章中更详细地讨论它。
\end{itemize}

\subsubsubsection{1.5.6\hspace{0.2cm}包的config文件}

CMake生态系统包括项目可以依赖的外部包,允许开发人员以无缝的、跨平台的方式使用库和工具。支持CMake的包应该提供一个配置文件,以便CMake理解如何使用它们。

我们将在第11章中学习如何编写这些文件。同时,这里有一些知识点需要记住:

\begin{itemize}
\item 
配置文件包含关于如何使用库二进制文件、头文件和辅助工具的信息。有时,它们会公开CMake宏,以便在项目中使用。

\item 
使用find\_package()指令来包含包。

\item 
描述包的CMake文件命名为<PackageName>-config.cmake和<PackageName>Config.cmake。

\item 
使用包时,可以指定需要的包的版本。CMake将检查相关的<Config>Version.cmake文件。

\item 
配置文件由支持CMake生态系统的包供应商提供。若供应商不提供这样的配置文件,可以用查找模块替换。

\item 
CMake提供了包注册表,用于在系统范围内为每个用户存储包。
\end{itemize}

\subsubsubsection{1.5.7\hspace{0.2cm}cmake\_install.cmake,CTestTestfile.cmake和CPackConfig.cmake}

这些文件是由生成阶段的cmake可执行文件在构建树中生成的,它们不应该通过手动编辑。CMake使用它们作为CMake安装操作、CTest和CPack的配置。若正在实现一个内源代码构建(不推荐),那么将它们添加到VCS的忽略文件中可能是个好主意。

\subsubsubsection{1.5.8\hspace{0.2cm}CMakePresets.json和CMakeUserPresets.json}

当需要明确缓存变量、所选择的生成器、构建树的路径等时,项目的配置可能会成为一项相对繁忙的任务——特别是当有多种构建项目的方法时。这就处于预设的用武之地了。

用户可以通过GUI选择预置,也可以使用命令行-{}-listpresets并使用-{}-preset=选项为构建系统选择预置。

预设值以相同的JSON格式存储在两个文件中:

\begin{itemize}
\item 
CMakePresets.json: 项目作者提供的预设。

\item 
CMakeUserPresets.json: 这是为希望根据自己的偏好定制项目配置用户准备的(可以将其添加到VCS的忽略文件中)。
\end{itemize}

预设是项目文件,然而在项目中并不是必需的,只有当我们完成初始设置时,预设才会有用。所以,若需要的话,可以跳到下一节:

\begin{lstlisting}[style=styleCMake]
# chapter-01/02-presets/CMakePresets.json
{
	"version": 1,
	"cmakeMinimumRequired": {
		"major": 3, "minor": 19, "patch": 3
	},
	"configurePresets": [ ],
	"vendor": {
		"vendor-one.com/ExampleIDE/1.0": {
			"buildQuickly": false
		}
	}
}
\end{lstlisting}

CMakePresets.json指定以下根字段:

\begin{itemize}
\item 
Version: 这是必需的,总是1。

\item 
cmakeMinimumRequired: 可选的。以带有三个字段的散列形式指定CMake版本:major、minor和patch。

\item 
vendor: IDE可以使用这个可选字段来存储它的元数据,其是一个以供应商域和斜杠分隔的路径为键的映射。CMake会忽略了这个字段。

\item 
configurePresets: 一个可选的可用预设数组。
\end{itemize}

让我们在configurePresets数组中添加两个预设值:

\begin{lstlisting}[style=styleCMake]
# chapter-01/02-presets/CMakePresets.json : my-preset
{
	"name": "my-preset",
	"displayName": "Custom Preset",
	"description": "Custom build - Ninja",
	"generator": "Ninja",
	"binaryDir": "${sourceDir}/build/ninja",
	"cacheVariables": {
		"FIRST_CACHE_VARIABLE": {
			"type": "BOOL", "value": "OFF"
		},
		"SECOND_CACHE_VARIABLE": "Ninjas rock"
	},
	"environment": {
		"MY_ENVIRONMENT_VARIABLE": "Test",
		"PATH": "$env{HOME}/ninja/bin:$penv{PATH}"
	},
	"vendor": {
		"vendor-one.com/ExampleIDE/1.0": {
			"buildQuickly": true
		}
	}
},
\end{lstlisting}

该文件支持树状结构,其中子预设从多个父预设继承属性。所以可以创建前面预设的副本,并且只覆盖我们需要的字段。下面是一个子预设的例子:

\begin{lstlisting}[style=styleCMake]
# chapter-01/02-presets/CMakePresets.json : my-preset-multi
{
	"name": "my-preset-multi",
	"inherits": "my-preset",
	"displayName": "Custom Ninja Multi-Config",
	"description": "Custom build - Ninja Multi",
	"generator": "Ninja Multi-Config"
}
\end{lstlisting}

\begin{tcolorbox}[colback=blue!5!white,colframe=blue!75!black,title=Note]
CMake文档只标记了一些明确需要的字段。但是,还有一些其他的字段标记为可选的,这些字段必须在预设中提供,或者从其父字段继承。
\end{tcolorbox}

预设定义为具有以下字段的映射:

\begin{itemize}
\item 
name: 标识预设的必需字符串,必须是机器友好,并在两个文件中具有唯一名称。

\item 
Hidden: 这是一个可选的布尔值,从GUI和命令行列表中隐藏预设值。这样的预设可以是另一个的父级,只需要提供名称。

\item 
displayName: 这是一个可选的字符串,提供人类友好的名称。

\item 
description: 这是一个描述预设的可选字符串。

\item 
Inherits: 这是要继承的预置名称的可选字符串或数组。在冲突的情况下,来自之前预设的值将是首选,并且每个预设都可以覆盖继承的字段。此外,CMakeUserPresets.json可以从项目预设继承,但不能从项目预设继承。

\item 
Vendor: 这是特定于供应商的值的可选映射,遵循与根级别供应商字段相同的约定。

\item 
Generator: 这是一个必需的或继承的字符串,用于指定要用于预设的生成器。

\item 
architecture和toolset: 这些是用于配置支持这些选项的生成器的可选字段(在生成项目构建系统部分中提到)。每个字段可以只是一个字符串或带有值和策略字段的哈希,其中策略是设置的或外部的。配置为set的strategy字段将设置该值,若生成器不支持该字段,则会产生错误。配置外部意味着为外部IDE设置了字段值,CMake应该忽略它。

\item 
binaryDir: 这是一个必需的或继承的字符串,提供到构建树目录的路径(该目录是绝对的或相对于源树的),支持宏展开。

\item 
cacheVariables: 这是一个可选的缓存变量映射,其中键表示变量名。可接受的值包括null、"TRUE"、"FALSE"、字符串值或带有可选类型字段和必选值字段的哈希。value可以是“TRUE”或“FALSE”的字符串值。缓存变量是通过联合操作继承的,除非将值指定为null,否则将保持未设置。字符串值支持宏展开。

\item 
Environment: 这是一个可选的环境变量映射,其中键表示变量名。接受的值包括空值或字符串值。除非将环境变量的值指定为null,否则将使用联合操作继承环境变量。字符串值支持宏展开,只要没有循环引用,变量可以以任何顺序相互引用。
\end{itemize}

识别和展开下列宏:

\begin{itemize}
\item 
\$\{sourceDir\}: 这是到源码树的路径。

\item 
\$\{sourceParentDir\}: 这是到源树的父目录的路径。

\item 
\$\{sourceDirName\}: 这是\$\{sourceDir\}的最后一个文件名组件。

例如,对于/home/rafal/project,其将是project。

\item 
\$\{presetName\}: 这是预设名称字段的值。

\item 
\$\{generator\}: 这是预设的生成器字段的值。

\item 
\$\{dollar\}: 这是一个美元符号(\$).

\item 
\$env\{<variable-name>\}: 这是一个环境变量宏。若定义了,将从预设值返回变量的值;否则,将从父环境返回值。预设中的变量名是区分大小写的(与Windows环境不同)。

\item 
\$penv\{<variable-name>\}: 这个选项类似于\$env,但总是返回来自父环境的值。这可以解决在预设环境变量中不允许使用循环引用的问题。

\item 
\$vendor\{<macro-name>\}: 供应商能够插入他们自己的宏。
\end{itemize}


\subsubsubsection{1.5.9\hspace{0.2cm}设置Git忽略文件}

目前有许多版本控制工具,其中最流行的一种是Git。每当我们开始一个新项目时,最好确保只签入需要存在的存储库文件。若只是向.gitignore文件中添加生成的、用户的或临时文件,那么保持项目健康就会更容易。通过这种方式,Git知道在构建新的提交时会自动跳过它们。以下是我在项目中使用的.gitignore文件:

\begin{lstlisting}[style=styleCMake]	
# chapter-01/01-hello/.gitignore
	
# If you put build tree in the source tree add it like so:
build_debug/
build_release/

# Generated and user files
**/CMakeCache.txt
**/CMakeUserPresets.json
**/CTestTestfile.cmake
**/CPackConfig.cmake
**/cmake_install.cmake
**/install_manifest.txt
**/compile_commands.json
\end{lstlisting}

项目中使用上述文件将为您和其他贡献者和用户提供更大的灵活性。

项目文件的未知领域现在已经逐渐显露。有了这个映射,就以编写自己的listfiles、配置缓存、准备预置等等。在开始项目编写之前,来看看还可以用CMake创建哪些其他类型的自包含单元。









