
CMake提供了许多脚本命令,允许使用变量和环境。其中一些在附录部分有详细介绍,例如list()、string()和file()。其他的,比如find\_…(),更适合在讨论依赖性管理的章节使用。

\subsubsubsection{2.6.1\hspace{0.2cm}message()指令}

我们已经了解message()指令,它其将文本打印到标准输出。然而,这其中的奥秘远比我们看到的要多得多。通过提供MODE参数,您可以自定义输出的样式,并且在出现错误的情况下,可以停止代码:message(<mode> "text")的执行。

可选的模式如下:

\begin{itemize}
\item 
FATAL\_ERROR: 这将停止处理和生成。

\item 
SEND\_ERROR: 这将继续处理,但跳过生成。

\item 
WARNING: 继续处理。

\item 
AUTHOR\_WARNING: CMake警告。继续处理。

\item 
DEPRECATION: 若启用了CMAKE\_ERROR\_DEPRECATED或CMAKE\_WARN\_DEPRECATED变量中的任何一个,将做出相应处理。

\item 
NOTICE或省略模式(默认): 这将向stderr输出一条消息,以吸引用户的注意。

\item 
STATUS: 这将继续处理,建议用于用户的主要消息。

\item 
VERBOSE: 这将继续处理,应该用于通常不是很有必要的更详细的信息。

\item 
DEBUG: 这将继续处理,并应该包含在项目出现问题时可能有用的详细信息。

\item 
TRACE: 这将继续处理,并建议在项目开发期间打印消息。通常,在发布项目之前,将这些类型的消息删除。
\end{itemize}

下面的例子在第一条消息之后停止执行:

\begin{lstlisting}[style=styleCMake]
# chapter02/10-useful/message_error.cmake

message(FATAL_ERROR "Stop processing")
message("Won't print this.")
\end{lstlisting}

根据当前日志级别(默认为STATUS)打印消息,在前一章的调试和跟踪选项部分讨论了如何更改这一点。这里就使用CMAKE\_MESSAGE\_CONTEXT进行调试,所以让我们开始吧。我们已经了解了调试的三个重要部分:列表、作用域和函数。

当启用命令行标志cmake -{}-log-context时,消息将装饰为点分隔的上下文,并存储在CMAKE\_MESSAGE\_CONTEXT列表中。考虑下面的例子:

\begin{lstlisting}[style=styleCMake]
# chapter02/10-useful/message_context.cmake
	
function(foo)
	list(APPEND CMAKE_MESSAGE_CONTEXT "foo")
	message("foo message")
endfunction()

list(APPEND CMAKE_MESSAGE_CONTEXT "top")
message("Before `foo`")
foo()
message("After `foo`")
\end{lstlisting}

前面脚本的输出如下所示:

\begin{tcblisting}{commandshell={}}
$ cmake -P message_context.cmake --log-context
[top] Before `foo`
[top.foo] foo message
[top] After `foo`
\end{tcblisting}

函数的初始作用域是从父作用域复制的,foo中的第一个命令在CMAKE\_MESSAGE\_CONTEXT中添加了一个foo函数名的新项。打印消息,函数作用域结束,丢弃本地复制的变量,并恢复之前的作用域(不含foo)。

这种方法对于复杂的目中的嵌套函数非常有用。希望永远都不需要它,但我认为这是一个很好的例子,说明了函数作用域在实践中是如何工作的。

message()的另一个很酷的技巧是在CMAKE\_MESSAGE\_INDENT列表中添加缩进(与CMAKE\_MESSAGE\_CONTEXT的方法完全相同):

\begin{lstlisting}[style=styleCMake]
list(APPEND CMAKE_MESSAGE_INDENT " ")
\end{lstlisting}

这样脚本的输出看起来就更清晰了:

\begin{tcblisting}{commandshell={}}
Before `foo`
  foo message
After `foo`
\end{tcblisting}

由于CMake没有提供带有断点或其他工具的真正调试器,当事情没有完全按计划进行时,干净的日志就非常重要了。

\subsubsubsection{2.6.2\hspace{0.2cm}include()指令}

我们可以将CMake代码划分到单独的文件中,以保持内容的有序和独立性。然后,可以通过include()从父列表文件引用:

\begin{lstlisting}[style=styleCMake]
include(<file|module> [OPTIONAL] [RESULT_VARIABLE <var>])
\end{lstlisting}

若提供文件名(一个扩展名为.cmake),CMake将尝试打开并执行它。注意,不会创建嵌套的、单独的作用域,因此对该文件中变量的任何更改都会影响调用作用域。

若文件不存在,CMake将抛出一个错误,除非我们用optional关键字指定是可选的。若需要知道include()是否成功,可以提供一个带有变量名的RESULT\_VARIABLE关键字。若成功,则用包含的文件的完整路径填充,失败则用未找到(NOTFOUND)填充。

脚本模式下运行时,将从当前工作目录解析相对路径。要强制搜索与脚本本身相关的内容,请提供绝对路径:

\begin{lstlisting}[style=styleCMake]
include("${CMAKE_CURRENT_LIST_DIR}/<filename>.cmake")
\end{lstlisting}

若不提供路径,但提供了模块的名称(没有.cmake或其他),CMake将尝试找到一个模块并包含它。然后,CMake将在CMake模块目录CMAKE\_MODULE\_PATH中,搜索名称为<module>.cmake的文件。

\subsubsubsection{2.6.3\hspace{0.2cm}include\_guard()指令}

包含有副作用的文件时,可能想要限制它们,以便它们只包含一次。这就是include\_guard([DIRECTORY|GLOBAL])的作用。

将include\_guard()放在包含的文件的顶部。当CMake第一次遇到它时,将在当前作用域中进行记录。若文件再次包含(可能是因为没有控制项目中的所有文件),将不会处理。

若想要防止在不相关的函数作用域中包含不会彼此共享变量的函数,应该提供DIRECTORY或GLOBAL参数。顾名思义,DIRECTORY关键字将在当前目录及其以下应用保护,而GLOBAL关键字将对整个构建应用保护。

\subsubsubsection{2.6.4\hspace{0.2cm}file()指令}

为了了解CMake脚本可以做什么,快速的浏览一下文件操作命令:

\begin{lstlisting}[style=styleCMake]
file(READ <filename> <out-var> [...])
file({WRITE | APPEND} <filename> <content>...)
file(DOWNLOAD <url> [<file>] [...])
\end{lstlisting}

简而言之,file()指令会以一种与系统无关的方式读取、写入和传输文件,并使用文件系统、文件锁、路径和存档。详情请参阅附录部分。

\subsubsubsection{2.6.5\hspace{0.2cm}execute\_process()指令}

有时,需要使用系统中可用的工具(毕竟,CMake主要是一个构建系统生成器)。CMake为此提供了一个命令:可以使用execute\_process()来运行其他进程并收集它们的输出。这个命令非常适合脚本,也可以在配置阶段的项目中使用。下面是命令的一般形式:

\begin{lstlisting}[style=styleCMake]
execute_process(COMMAND <cmd1> [<arguments>]… [OPTIONS])
\end{lstlisting}

CMake将使用操作系统的API来创建子进程(因此,诸如\&\&、||和>等shell操作符将不起作用)。但是,可以通过不止一次地提供COMMAND <cmd> <arguments>参数来连接命令,并将一个命令的输出传递给另一个命令。

若进程没有在要求的限制内完成任务,可以选择使用TIMEOUT <seconds>参数来终止进程,并且可以根据需要设置WORKING\_DIRECTORY <directory>。

通过RESULTS\_VARIABLE <variable>参数,可以在列表中收集所有任务的退出代码。若只对最后执行命令的结果感兴趣,请使用单数形式:result \_VARIABLE <variable>。

为了收集输出,CMake提供了两个参数:OUTPUT\_VARIABLE和ERROR\_VARIABLE(以类似的方式使用)。若想合并stdout和stderr,请对两个参数使用相同的变量。

记住,在为其他用户编写项目时,应该确保计划使用的命令在平台上可用。
















