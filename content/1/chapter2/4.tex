要存储一个列表,CMake将所有元素连接成一个字符串,使用分号(;)作为分隔符:a;list;of;5;elements。可以用反斜杠转义元素中的分号,像这样: a\verb|\|;single\verb|\|;element.

要创建一个列表,可以使用set()指令:set(myList一个包含五个元素的列表)。由于列表的存储方式不同,下面的命令将具有完全相同的效果:

\begin{itemize}
\item 
set(myList "a;list;of;five;elements")

\item 
set(myList a list "of;five;elements")
\end{itemize}

CMak会e自动解包未加引号的参数中的列表。通过传递一个不加引号的myList引用,可以有效地向指令传递更多的参数:

\begin{lstlisting}[style=styleCMake]
message("the list is:" ${myList})
\end{lstlisting}

message()指令将在这里接收6个参数:“list is:”,“a”,“list”,“of”,“five”,“elements”。这可能会产生意想不到的后果,因为输出将在参数之间没有任何空格的情况下打印出来:

\begin{tcblisting}{commandshell={}}
the list is:alistoffiveelements
\end{tcblisting}

这是一个非常简单的机制,应该谨慎使用。

CMake提供了一个list()指令,该指令提供了大量子命令来读取、搜索、修改和排序列表。下面是其中一部分:

\begin{lstlisting}[style=styleCMake]
list(LENGTH <list> <out-var>)
list(GET <list> <element index> [<index> ...] <out-var>)
list(JOIN <list> <glue> <out-var>)
list(SUBLIST <list> <begin> <length> <out-var>)
list(FIND <list> <value> <out-var>)
list(APPEND <list> [<element>...])
list(FILTER <list> {INCLUDE | EXCLUDE} REGEX <regex>)
list(INSERT <list> <index> [<element>...])
list(POP_BACK <list> [<out-var>...])
list(POP_FRONT <list> [<out-var>...])
list(PREPEND <list> [<element>...])
list(REMOVE_ITEM <list> <value>...)
list(REMOVE_AT <list> <index>...)
list(REMOVE_DUPLICATES <list>)
list(TRANSFORM <list> <ACTION> [...])
list(REVERSE <list>)
list(SORT <list> [...])
\end{lstlisting}

大多数时候,我们并不需要在项目中使用列表。但会发现在极少数情况下这个概念是方便的,可以在附录部分中找到list()指令的更多信息。

现在已经知道了如何使用各种类型的列表和变量,让我们将注意力转移到控制执行流上,并学习CMake中可用的控制结构。



















