本章开启了使用CMake进行实际编程的大门——您现在能够编写出色的、信息丰富的注释和调用内置指令,并且了解如何正确地为它们提供各种参数。仅这些知识就可以帮助您理解在其他项目中可能见过的CMake列表文件的不寻常语法。

接下来,我们介绍了CMake中的变量——具体来说,如何引用、设置和取消设置普通变量、缓存变量和环境变量。我们深入研究了目录和函数作用域的工作方式,并讨论了与嵌套作用域相关的问题(及其解决方法)。

还讨论了列表和控制结构,讨论了条件的语法、逻辑操作、无引号求值,以及字符串和变量。我们学习了如何比较值,进行简单的检查,以及检查系统中文件的状态。这样就编写条件块和while循环。当我们讨论循环的时候,也掌握了foreach循环的语法。

我相信,知道如何用宏和函数语句定义自己的命令,将有助于您以更过程化的风格编写更清晰的代码。我们还分享了一些关于如何更好地构造我们的代码并提出更可读的名称的想法。

最后,正式介绍了message(),及其多个日志级别。还研究了如何划分和包含列表文件,并发现了一些其他有用的命令。了解了这些,就可以进入下一章了,在CMake中编写我们的第一个项目。