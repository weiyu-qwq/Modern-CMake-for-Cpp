在第1章中,讨论了源代码内构建,以及如何建议始终将构建路径指定为源代码外的。这不仅可以具有更干净的构建树和更简单的.gitignore文件,而且还减少了不小心覆盖或删除源码文件的可能。

在网上搜索解决方案时,可能会发现StackOverflow中有相同的问题:\url{https://stackoverflow.com/q/1208681/6659218}。作者注意到,无论做什么,CMake似乎仍然会创建一个CMakeFiles/目录和一个CMakeCache.txt文件。一些答案建议使用无文档变量,确保用户不能在源目录中写入:

\begin{lstlisting}[style=styleCMake]
# add this options before PROJECT keyword
set(CMAKE_DISABLE_SOURCE_CHANGES ON)
set(CMAKE_DISABLE_IN_SOURCE_BUILD ON)
\end{lstlisting}

我想说,在使用任何软件的无文档特性时都要谨慎,因为它们可能会在没有警告的情况下消失。CMake 3.20中设置上述变量会终止构建并产生一个相当难看的错误:

\begin{tcblisting}{commandshell={}}
CMake Error at /opt/cmake/share/cmake-3.20/Modules/
CMakeDetermineSystem.cmake:203 (file):
  file attempted to write a file:
  /root/examples/chapter03/09-in-source/CMakeFiles/CMakeOutput.
log into a source directory.
\end{tcblisting}

然而,它仍然创建了上述文件!因此,建议是使用一个更老的——但得到充分支持的——机制:

\begin{lstlisting}[style=styleCMake]
# chapter03/09-in-source/CMakeLists.txt

cmake_minimum_required(VERSION 3.20.0)
project(NoInSource CXX)
if(PROJECT_SOURCE_DIR STREQUAL PROJECT_BINARY_DIR)
	message(FATAL_ERROR "In-source builds are not allowed")
endif()
message("Build successful!")
\end{lstlisting}

若Kitware (CMake背后的公司)决定正式支持CMAKE\_DISABLE\_SOURCE\_CHANGES或CMAKE\_DISABLE\_IN\_SOURCE\_BUILD,那么可以切换到那种解决方案。














