In Chapter 1, First Steps with CMake, we talked about in-source builds, and how it is recommended to always specify the build path to be out-of-source. This not only allows for a cleaner build tree and a simpler .gitignore file, but it also decreases the chances you'll accidentally overwrite or delete any source files.

Searching for the solution online, you may stumble on a StackOverflow thread that asks the same question: \url{https://stackoverflow.com/q/1208681/6659218}. Here, the author notices that no matter what you do, it seems like CMake will still create a CMakeFiles/ directory and a CMakeCache.txt file. Some answers suggest using undocumented variables to make sure that the user can't write in the source directory under any circumstances:

\begin{lstlisting}[style=styleCMake]
# add this options before PROJECT keyword
set(CMAKE_DISABLE_SOURCE_CHANGES ON)
set(CMAKE_DISABLE_IN_SOURCE_BUILD ON)
\end{lstlisting}

I'd say to be cautious when using undocumented features of any software, as they may go away without warning. Setting the preceding variables in CMake 3.20 terminates the build with a rather ugly error:

\begin{tcblisting}{commandshell={}}
CMake Error at /opt/cmake/share/cmake-3.20/Modules/
CMakeDetermineSystem.cmake:203 (file):
  file attempted to write a file:
  /root/examples/chapter03/09-in-source/CMakeFiles/CMakeOutput.
log into a source
  directory.
\end{tcblisting}

However, it still creates the mentioned files anyway! Therefore, my recommendation is to go with an older – but fully supported – mechanism:

\begin{lstlisting}[style=styleCMake]
# chapter03/09-in-source/CMakeLists.txt

cmake_minimum_required(VERSION 3.20.0)
project(NoInSource CXX)
if(PROJECT_SOURCE_DIR STREQUAL PROJECT_BINARY_DIR)
	message(FATAL_ERROR "In-source builds are not allowed")
endif()
message("Build successful!")
\end{lstlisting}

If Kitware (company behind the CMake) ever decides to officially support CMAKE\_DISABLE\_SOURCE\_CHANGES or CMAKE\_DISABLE\_IN\_SOURCE\_BUILD, then by all means, switch to that solution.














