
CMake通过CMAKE\_变量、ENV变量和特殊命令提供了多种查询环境的方法。例如,收集的信息可以用于支持跨平台脚本。这些机制可避免使用特定于平台的shell命令,这些命令可能不容易移植,或者在不同的环境中命名不同。

对于性能关键型应用程序,了解目标平台的所有特性(例如,指令集、CPU核心数等)将非常有用。然后,可以将此信息传递给已编译的二进制文件,以便对期进行优化(将在下一章学习如何做到这一点)。先来看看原生CMake中有哪些信息可用。

\subsubsubsection{3.5.1\hspace{0.2cm}识别操作系统}

很多情况下,了解目标操作系统是很有用的。即使是像文件系统这样普通的东西,Windows和Unix在大小写敏感、文件路径结构、扩展名、特权等方面也有很大的不同。系统上的大多数命令在另一个系统上是不可用的,或者可以以不同的方式命名(即使是用一个字母命名——例如,ifconfig和ipconfig命令)。

若需要用一个CMake脚本支持多个目标操作系统,只要检查CMAKE\_SYSTEM\_NAME变量,就可以采取相应的行动。这里有一个简单的例子:

\begin{lstlisting}[style=styleCMake]
if(CMAKE_SYSTEM_NAME STREQUAL "Linux")
	message(STATUS "Doing things the usual way")
elseif(CMAKE_SYSTEM_NAME STREQUAL "Darwin")
	message(STATUS "Thinking differently")
elseif(CMAKE_SYSTEM_NAME STREQUAL "Windows")
	message(STATUS "I'm supported here too.")
elseif(CMAKE_SYSTEM_NAME STREQUAL "AIX")
	message(STATUS "I buy mainframes.")
else()
	message(STATUS "This is ${CMAKE_SYSTEM_NAME} speaking.")
endif()
\end{lstlisting}

有一个包含操作系统版本的变量:CMAKE\_SYSTEM\_VERSION。然而,建议是尽量使解决方案与系统无关,并使用内置的CMake跨平台功能。特别是对于文件系统上的操作,应该使用file()指令。

\subsubsubsection{3.5.2\hspace{0.2cm}交叉编译——主机系统和目标系统?}

在一台机器上编译要在另一台机器上运行的代码,称为交叉编译。可以(使用正确的工具集)通过在Windows机器上运行CMake来编译Android应用程序。交叉编译不在本书的讨论范围内,但是理解它如何影响CMake的某些部分是很重要的。

允许交叉编译的必要步骤之一是将CMAKE\_SYSTEM\_NAME和CMAKE\_SYSTEM\_VERSION变量设置为适合为目标编译的操作系统的值(CMAKE文档将其称为目标系统)。用于执行构建的操作系统称为主机系统。

无论如何配置,主机系统上的信息总是可以在名称中带有HOST关键字的变量中访问:CMAKE\_HOST\_SYSTEM\_NAME、CMAKE\_HOST\_SYSTEM\_PROCESSOR和CMAKE\_HOST\_SYSTEM\_VERSION。

还有一些变量的名称中带有HOST关键字,因此请记住它们显式引用主机系统。否则,所有变量都引用目标系统(通常是宿主系统,除非交叉编译)。

若有兴趣阅读更多关于交叉编译的内容,建议您参考CMake文档:\url{https://cmake.org/cmake/help/latest/manual/ cmake-toolchains.7.html}。

\subsubsubsection{3.5.3\hspace{0.2cm}简化变量}

CMake将预定义一些变量,这些变量将提供关于主机和目标系统的信息。若使用特定的系统,则将适当的变量设置为非false值(即1或true):

\begin{itemize}
\item 
ANDROID, APPLE, CYGWIN, UNIX, IOS, WIN32, WINCE, WINDOWS\_PHONE

\item 
CMAKE\_HOST\_APPLE, CMAKE\_HOST\_SOLARIS, CMAKE\_HOST\_UNIX, CMAKE\_HOST\_WIN32
\end{itemize}

对于32和64位版本的Windows和MSYS,WIN32和CMAKE\_HOST\_WIN32变量将为真(此值为遗留原因保留)。此外,UNIX将适用于Linux、macOS和Cygwin。

\subsubsubsection{3.5.4\hspace{0.2cm}主机系统信息}

CMake可以提供更多变量,但为了节省时间,不会查询环境中很少需要的信息,比如处理器是否支持MMX或总物理内存是多少。这并不意味着这个信息不可用——只需要用下面的命令显式地进行:

\begin{lstlisting}[style=styleCMake]
cmake_host_system_information(RESULT <VARIABLE> QUERY <KEY>…)
\end{lstlisting}

需要提供一个目标变量和感兴趣的键的列表。若只提供一个键,则变量将只包含一个值;否则,它将是一个值列表。可以询问很多关于环境和操作系统的细节:

\begin{table}[H]
	\begin{tabular}{|l|l|}
		\hline
		\textbf{键值}                & \textbf{返回值}                              \\ \hline
		HOSTNAME                    & 主机名                                      \\ \hline
		FQDN                        & 完全限定的dimian名称                   \\ \hline
		TOTAL\_VIRTUAL\_MEMORY      & 总虚拟内存(MiB)                    \\ \hline
		AVAILABLE\_VIRTUAL\_MEMORY  & 可用的虚拟内存(MiB)                \\ \hline
		TOTAL\_PHYSICAL\_MEMORY     & 总物理内存(MiB)                   \\ \hline
		AVAILABLE\_PHYSICAL\_MEMORY & 可用的物理内存(MiB)               \\ \hline
		OS\_NAME &
		\begin{tabular}[c]{@{}l@{}}如果该命令存在,则uname -s的输出;若没有,\\ 则uname -s是Windows、Linux或Darwin中的一个\end{tabular} \\ \hline
		OS\_RELEASE                 & OS子类型,例如Windows Professional \\ \hline
		OS\_VERSION                 & 操作系统构建ID                                    \\ \hline
		OS\_PLATFORM &
		\begin{tabular}[c]{@{}l@{}}在Windows上,返回的是 \\ PROCESSOR\_ARCHITECTURE环境变量。在Unix和\\ macOS上,该值包含uname -m命令的输出。\end{tabular} \\ \hline
	\end{tabular}
\end{table}

可以查询特定于处理器的信息:

\begin{table}[H]
	\begin{tabular}{|l|l|}
		\hline
		\textbf{键值}                & \textbf{返回值}                          \\ \hline
		NUMBER\_OF\_LOGICAL\_CORES  & 逻辑核数                      \\ \hline
		NUMBER\_OF\_PHYSICAL\_CORES & 物理核数                     \\ \hline
		HAS\_SERIAL\_NUMBER         & 若处理器有序列号,则为1           \\ \hline
		PROCESSOR\_SERIAL\_NUMBER   & 处理器序列号                      \\ \hline
		PROCESSOR\_NAME             & 可读的处理器名称                \\ \hline
		PROCESSOR\_DESCRIPTION      & 可读的完整处理器描述    \\ \hline
		IS\_64BIT                   & 若处理器是64位,则为1                     \\ \hline
		HAS\_FPU                    & 处理器有浮点单元,则返回1     \\ \hline
		HAS\_MMX                    & 若处理器支持MMX指令,则返回1     \\ \hline
		HAS\_MMX\_PLUS              & 若处理器支持Ext. MMX指令,则返回1 \\ \hline
		HAS\_SSE                    & 若果处理器支持SSE指令 ,则返回1     \\ \hline
		HAS\_SSE2                   & 若处理器支持SSE2指令,则返回1     \\ \hline
		HAS\_SSE\_FP                & 若处理器支持SSE FP指令,则返回1   \\ \hline
		HAS\_SSE\_MMX               & 若处理器支持SSE MMX指令,则返回1  \\ \hline
		HAS\_AMD\_3DNOW             & 若处理器支持3DNow指令,则返回1    \\ \hline
		HAS\_AMD\_3DNOW\_PLUS       & 若处理器支持3DNow+指令,则返回1   \\ \hline
		HAS\_IA64                   & 若是模拟x86的IA-64处理器,则返回1      \\ \hline
	\end{tabular}
\end{table}

\subsubsubsection{3.5.5\hspace{0.2cm}平台是32位还是64位架构?}

64位架构中,内存地址、处理器寄存器、处理器指令、地址总线和数据总线都是64位宽的。虽然这是一个简化的定义,但说明了64位平台与32位平台的不同之处。

C++中,不同的架构意味着一些基本数据类型(int和long)和指针的位宽不同。CMake利用指针的大小来收集关于目标计算机的信息。这个信息可以通过CMAKE\_SIZEOF\_VOID\_P变量获得,它将包含64位的值8(因为指针是8字节宽的)和32位的值4(4字节):

\begin{lstlisting}[style=styleCMake]
if(CMAKE_SIZEOF_VOID_P EQUAL 8)
	message(STATUS "Target is 64 bits")
endif()
\end{lstlisting}

\subsubsubsection{3.5.6\hspace{0.2cm}系统的端序}

架构可以是大端的,也可以是小端的。端序是字的字节顺序或处理器的自然数据单位。大端存储系统将最高位字节存储在最低内存地址,将最低位字节存储在最高内存地址。小端系统正好相反。

大多数情况下,字节顺序并不重要,但当编写需要可移植的位代码时,CMake将为你提供一个BIG\_ENDIAN或LITTLE\_ENDIAN值存储在CMAKE\_<lang>\_BYTE\_ORDER变量中,其中<lang>为C、CXX、OBJC或CUDA。

知道了如何查询环境,现在让我们将注意力转移到项目的关键设置上。

