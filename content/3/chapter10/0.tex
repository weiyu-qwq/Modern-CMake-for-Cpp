高质量的代码不仅要写得很好、工作良好、测试良好,而且还要有完整的文档记录。文档允许我们共享可能丢失的信息,描绘更大的图景,提供上下文,揭示意图,并最终让外部用户和维护人员更容易理解代码或项目。

还记得上次加入的新项目么?在迷宫般的目录和文件中迷失了好几个小时吗?这本可以避免的。真正优秀的文档可以让新手在几秒钟内找到他们想要的代码行。遗憾的是,缺少文档的问题常常被掩盖起来。这并不奇怪,这需要很多技巧,许多人并不擅长制作文档。最重要的是,文档和代码工作是分离的,除非制定了严格的更新和审查流程,否则很容易忘记更新文档。

有些团队(为了节省时间或受到管理人员的鼓励)遵循编写“自文档化代码”的实践。通过为文件名、函数、变量等选择有意义的、可读的标识符,从而避免记录的繁琐工作。虽然好的命名习惯是绝对正确的,但其不能取代文档。即使是最好的函数签名也不能保证传递所有必要的信息——例如,int removeduplates ();是非常描述性的,但并没有揭示返回的是什么!可能是找到的副本数量,剩下的物品数量,或者其他东西——这并不确定。记住:天下没有免费的午餐。

为了简化工作,专业人员使用自动文档生成器,它可以分析源文件中的代码和注释,以多种不同格式生成全面的文档。在CMake项目中添加这样的生成器非常简单——来看看文档是怎么生成!

本章中,我们将讨论以下主题:

\begin{itemize}
\item 
添加Doxygen

\item 
生成好看的文档
\end{itemize}
