
上一节证明了编写一个小型单元测试驱动程序并不复杂,但专业开发人员实际上喜欢重新发明轮子(会比传统的更漂亮、更圆、更快)。不要落入这个陷阱:这样会创建太多的样板文件,可能会成为一个独立的项目。在解决方案中引入流行的单元测试框架,使其与超越项目和公司的标准保持一致,可以以低廉的成本提供免费的更新和扩展。你,值得拥有!

如何将单元测试框架添加到项目中?根据所选框架的规则在实现文件中编写测试,并测试与框架提供的测试运行器链接起来。测试运行程序是将开始执行选定测试的入口点。与前面看到的unit\_tests.cpp文件不同,它们中的许多将自动检测所有测试。很漂亮!

我会在本章中介绍两个单元测试框架,选择他们的原因如下所示:

\begin{itemize}
\item 
Catch2是一个相对容易学习、支持良好且有文档记录的项目。提供了简单的测试用例,但也为行为驱动开发(BDD)提供了优雅的宏。但缺少一些功能,可以在需要时与外部工具结合使用。可以浏览其主页:\url{https://github.com/catchorg/Catch2}。

\item 
GTest也非常方便,但更高级。主要特性是具有丰富的断言、用户定义的断言、死亡测试、致命和非致命失败、值和类型参数化测试、XML测试报告生成和模拟。最后一个是在GMock模块中交付的,可从相同的存储库获得:\url{https://github.com/google/googletest}。
\end{itemize}

应该选择哪种框架取决于自己的偏好和项目的规模。若更喜欢一个缓慢、渐进的过程,不需要那些花哨的东西,那就用Catch2吧。喜欢从深层入手,并有大量开发人员可用的话,可以选择GTest。

\subsubsubsection{8.5.1\hspace{0.2cm}Catch2}

这由Martin维护的框架Hořeňovský非常适合初学者和较小的项目。这并不是说它不能处理更大的应用程序,在某些情况可能需要额外的工具。若详细地讲下去,就会偏离本书的主题,但还是简单进行一下概述。首先,可以为Calc类编写单元测试的实现:

\begin{lstlisting}[style=styleCXX]
//chapter08/03-catch2/test/calc_test.cpp

#include <catch2/catch_test_macros.hpp>

#include "calc.h"
TEST_CASE("SumAddsTwoInts", "[calc]") {
	Calc sut;
	CHECK(4 == sut.Sum(2, 2));

}
TEST_CASE("MultiplyMultipliesTwoInts", "[calc]") {
	Calc sut;
	CHECK(12 == sut.Multiply(3, 4));
}
\end{lstlisting} 

这几行代码比前面的例子中的测试强大得多。CHECK()宏不仅会验证是否满足预期——实际上会收集所有失败的断言,并显示在单个输出中,以便可以进行修复,并避免重新编译。

现在,不需要添加这些测试,甚至不需要通知CMake;可以忘记add\_test(),Catch2将自动向CTest注册测试。在配置了项目之后,添加框架非常容易,只需要使用FetchContent()将其引入项目即可。

有两个主要版本可供选择:v2和v3。版本2作为C++11的单头库(仅\#include <catch2/catch.hpp>)提供,最终将在版本3中弃用。它有多个头文件,可以编译为静态库,需要支持C++14。当然,若还没使用现代C++(C++11不再认为是“现代的”)编译器,建议使用更新的版本。使用Catch2时,应该选择一个Git提交,并将其固定在列表文件中。换句话说,这里不能保证升级不会破坏测试的代码(很可能不会,但若不需要,就不要冒险使用devel分支)。为了获取Catch2,需要向存储库提供一个URL:

\begin{lstlisting}[style=styleCMake]
# chapter08/03-catch2/test/CMakeLists.txt
include(FetchContent)
FetchContent_Declare(
	Catch2
	GIT_REPOSITORY https://github.com/catchorg/Catch2.git
	GIT_TAG v3.0.0
)
FetchContent_MakeAvailable(Catch2)
\end{lstlisting} 

然后,定义unit\_tests目标,并将其与sut和框架提供的入口点,以及Catch2::Catch2WithMain库链接起来。因为Catch2提供了main(),所以这里不再使用unit\_tests.cpp文件(这个文件可以删除)。代码如下所示:

\begin{lstlisting}[style=styleCMake]
# chapter08/03-catch2/test/CMakeLists.txt (continued)

add_executable(unit_tests
			calc_test.cpp
			run_test.cpp)
target_link_libraries(unit_tests PRIVATE
						sut Catch2::Catch2WithMain)
\end{lstlisting} 

最后,使用catch\_discover\_tests()指令,在Catch2提供的模块中,将检测unit\_tests中所有测试用例,并将它们注册到CTest:

\begin{lstlisting}[style=styleCMake]
#chapter08/03-catch2/test/CMakeLists.txt (continued)

list(APPEND CMAKE_MODULE_PATH ${catch2_SOURCE_DIR}/extras)
include(Catch)
catch_discover_tests(unit_tests)
\end{lstlisting} 

完成了!我们已经在解决方案中添加了一个单元测试框架。现在来看看在实践中的情况。测试运行器的输出:

\begin{tcblisting}{commandshell={}}
# ./test/unit_tests
unit_tests is a Catch v3.0.0 host application.
Run with -? for options
--------------------------------------------------------------
MultiplyMultipliesTwoInts
--------------------------------------------------------------
examples/chapter08/03-catch2/test/calc_test.cpp:9
..............................................................
examples/chapter08/03-catch2/test/calc_test.cpp:11: FAILED:
  CHECK( 12 == sut.Multiply(3, 4) )
with expansion:
  12 == 9
==============================================================
test cases: 3 | 2 passed | 1 failed
assertions: 3 | 2 passed | 1 failed
\end{tcblisting}

直接执行运行程序(已编译的unit\_test可执行文件)稍微快一些,通常情况下,应该使用ctest -{}-output-on-failure,而不是直接执行测试运行程序,以获得前面提到的所有ctest好处。注意,Catch2能够方便地扩展sut。

这就是Catch2的设置。若需要添加更多的测试,只需创建实现文件,并将其路径插入到unit\_tests目标的源列表中即可。

这个框架有很多有趣的技巧:事件监听器、数据生成器和微基准测试,但没有提供模拟功能。若不知道mock是什么的读者,请继续阅读。然而,若发现自己需要模拟,可以在Catch2旁边添加一个模拟框架:

\begin{itemize}
\item 
FakeIt (\url{https://github.com/eranpeer/FakeIt})

\item 
Hippomocks (\url{https://github.com/dascandy/hippomocks})

\item 
Trompeloeil (\url{https://github.com/rollbear/trompeloeil})
\end{itemize}

对于更精简、更高级的体验,还有另一个框架值得了解。

\subsubsubsection{8.5.2\hspace{0.2cm}GTest}

使用GTest有几个优点:在C++社区中得到了高度认可(多个IDE本身就支持)。这个星球上最大的搜索引擎公司在维护和使用它,所以其不太可能在短时间内变得过时或抛弃。可以测试C++11及以上版本,所以若使用旧一些的环境,也不必担心。

GTest存储库包含两个项目:GTest(主要测试框架)和GMock(一个添加了模拟功能的库),可以通过FetchContent()来下载。

\hspace*{\fill} \\ %插入空行
\noindent
\textbf{使用GTest}

项目需要遵循结构化测试的指导,下面是如何在这个框架中编写单元测试:

\begin{lstlisting}[style=styleCXX]
// chapter08/04-gtest/test/calc_test.cpp

#include <gtest/gtest.h>
#include "calc.h"

class CalcTestSuite : public ::testing::Test {
	protected:
	Calc sut_;
};

TEST_F(CalcTestSuite, SumAddsTwoInts) {
	EXPECT_EQ(4, sut_.Sum(2, 2));
}

TEST_F(CalcTestSuite, MultiplyMultipliesTwoInts) {
	EXPECT_EQ(12, sut_.Multiply(3, 4));
}
\end{lstlisting} 

这个例子也将在GMock中使用,可以将测试放在单独的CalcTestSuite类中。测试套件是与组相关的测试,可以重用相同的字段、方法、设置(初始化)和拆卸(清理)步骤。要创建测试套件,需要声明继承自::testing::Test的新类,并将重用的元素(字段、方法)放在其protected部分中。

测试套件中的每个测试用例都用TEST\_F()预处理器宏声明,该宏将测试套件和测试用例提供的名称字符串化(还有简单的TEST()宏,定义了不相关的测试)。因为在类中定义了Calc sut\_,所以每个测试用例都可以访问,就好像测试是CalcTestSuite的一个方法一样。实际上,每个测试用例都在自己的类中运行,隐式地继承自CalcTestSuite(这就是需要protected的原因)。注意,重用字段并需要在连续的测试之间共享数据——其功能是为了保持代码DRY。

GTest不像Catch2那样为断言提供自然的语法。而需要使用显式比较,例如EXPECT\_EQ()。通常,将期望值作为第一个参数,将实际值作为第二个参数。还有许多其他的断言、帮助器和宏值得学习。

\begin{tcolorbox}[colback=blue!5!white,colframe=blue!75!black,title=Note]
有关GTest的详细信息,请参阅官方参考资料(\url{https://google.github.io/googletest/})
\end{tcolorbox}

要将这个依赖项添加到我们的项目中,需要决定使用哪个版本。与Catch2不同,GTest倾向于“live at head”的理念(源自GTest所依赖的Abseil项目),其指出:“若从源代码构建依赖并遵循API,应该不会有问题。”(更多细节请参考扩展阅读部分。)

若遵循此规则(并且从源代码构建也不是问题),可以将Git标记设置为主分支。否则,从GTest存储库中选择一个版本。还可以选择首先在主机上搜索已安装的副本,因为CMake提供了一个绑定的FindGTest模块来查找本地安装。从3.20版本开始,CMake将使用上游的GTestConfig.cmake配置文件,将替换依赖于查找模块的方式。

GTest上添加依赖:

\begin{lstlisting}[style=styleCMake]
# chapter08/04-gtest/test/CMakeLists.txt

include(FetchContent)
FetchContent_Declare(
	googletest
	GIT_REPOSITORY https://github.com/google/googletest.git
	GIT_TAG master
)
set(gtest_force_shared_crt ON CACHE BOOL "" FORCE)
FetchContent_MakeAvailable(googletest)
\end{lstlisting}

遵循与Catch2执行FetchContent()的方法,并从源代码构建框架。唯一的区别是添加了set(gtest…),这是gtest作者推荐的,以防止在Windows上覆盖父项目的编译器和链接器设置。

最后,可以声明可执行的测试运行器,并链接到gtest\_main,并通过内置的CMake GoogleTest模块自动找到我们的测试用例:

\begin{lstlisting}[style=styleCMake]
# chapter08/04-gtest/test/CMakeLists.txt (continued)
	
add_executable(unit_tests
				calc_test.cpp
				run_test.cpp)
target_link_libraries(unit_tests PRIVATE sut gtest_main)
include(GoogleTest)
gtest_discover_tests(unit_tests)
\end{lstlisting}

这就完成了GTest的设置。测试运行器的输出比Catch2的输出要详细得多,可以通过-{}-gtest\_brief=1将其限制为仅适用于失败用例:

\begin{tcblisting}{commandshell={}}
# ./test/unit_tests --gtest_brief=1
~/examples/chapter08/04-gtest/test/calc_test.cpp:15: Failure
Expected equality of these values:
  12
  sut_.Multiply(3, 4)
    Which is: 9
[ FAILED ] CalcTestSuite.MultiplyMultipliesTwoInts (0 ms)
[==========] 3 tests from 2 test suites ran. (0 ms total)
[ PASSED ] 2 tests.
\end{tcblisting}

从CTest运行时,会抑制噪声输出(除非使用ctest -{}-output-on-failure显式地启用)。

现在已经有了框架,再来讨论一下mock。毕竟,当测试与其他元素结合在一起时,不可能是真正的“单元”。

\subsubsubsection{8.5.3\hspace{0.2cm}GMock}

编写真正的单元测试是要与其他代码隔离使执行的代码,这样的单元可以理解为是一个自包含的元素,要么是类,要么是组件。当然,若没有用C++编写的程序,那么所有单元都是与其他单元隔离。最有可能的是,代码将严重依赖于类之间的某种形式的关系。这样做只有一个问题:类对象将需要另一个类对象,而另一个类又需要另一个类对象;从而,整个解决方案都参与了“单元测试”。更糟糕的是,代码可能与外部系统耦合,并依赖于它的状态——例如,数据库中的特定记录、传入的网络数据包,或存储在磁盘上的特定文件。

为了解耦单元以进行测试,开发人员使用测试副本或测试类使用的特殊版本的类。例子包括fake,stub和mock。以下是一些粗略的定义:

\begin{itemize}
\item 
fake是一些更复杂类的有限实现。一个例子可能是内存中的映射,而不是实际的数据库客户端。

\item 
stub为方法调用提供特定的、固定的回答,仅限于测试使用的响应。还可以记录调用了哪些方法,以及调用了多少次。

\item 
mock是stub的扩展版本,其还将验证在测试期间是否按预期调用了方法。
\end{itemize}

这样的测试双精度对象在测试开始时创建,并作为参数提供给测试类的构造函数,以代替实际对象。这种机制称为“依赖注入”。简单测试双精度的问题是它们太简单了。为了模拟不同测试场景的行为,必须提供许多不同的double,一个用于耦合对象可能出现的状态。这不是很现实,而且会将测试代码分散到太多的文件中。这就是GMock的用武之地了:允许开发人员为特定的类创建通用测试副本,并为每行测试定义其行为。GMock称这些双精度对象为“mock”,但它们是上述所有类型的混合,具体取决于场合。

考虑下面的例子:向Calc类添加一个函数,向提供的参数添加一个随机数。其由AddRandomNumber()方法表示,该方法将这个和作为int返回。如何确认返回值确实是某个随机值和提供给类的值的精确和?依赖随机性是许多重要过程的关键,若不正确地使用,可能会有很严重的后果。检查所有的随机数,直到穷尽所有是不可能的。

为了测试,需要在一个可以模拟的类中包装一个随机数生成器(或者,用模拟替换)。mock允许强制一个特定的响应,即“假”生成一个随机数。Calc将在AddRandomNumber()中使用该值,并允许检查从该方法返回的值是否符合预期。随机数生成与另一个单元的清晰分离是一种方式(因为可以将一种类型的生成器交换为另一种类型的生成器)。

从抽象生成器的公共接口开始,可以在实际的生成器和mock中实现,使我们能够交替使用。我们将执行以下代码:

\begin{lstlisting}[style=styleCXX]
// chapter08/05-gmock/src/rng.h

#pragma once
class RandomNumberGenerator {
	public:
	virtual int Get() = 0;
	virtual ~RandomNumberGenerator() = default;
};
\end{lstlisting}

实现此接口的类将为我们提供来自Get()方法的随机数。注意虚拟关键字——必须在所有需要模拟的方法上,除非想涉及更复杂的基于模板的模拟。还需要记住添加一个虚析构函数。接下来,必须扩展Calc类来接受和存储生成器:

\begin{lstlisting}[style=styleCXX]
// chapter08/05-gmock/src/calc.h

#pragma once
#include "rng.h"

class Calc {
	RandomNumberGenerator* rng_;
public:
	Calc(RandomNumberGenerator* rng);
	int Sum(int a, int b);
	int Multiply(int a, int b);
	int AddRandomNumber(int a);
};
\end{lstlisting}

包含了头文件并添加了一个方法来提供随机添加,还创建了一个用于存储指向生成器的指针的字段,以及一个参数化的构造函数。这就是依赖注入在实践中的工作方式。实现了这些方法:

\begin{lstlisting}[style=styleCXX]
// chapter08/05-gmock/src/calc.cpp

#include "calc.h"

Calc::Calc(RandomNumberGenerator* rng) {
	rng_ = rng;
}

int Calc::Sum(int a, int b) {
	return a + b;
}

int Calc::Multiply(int a, int b) {
	return a * b; // now corrected
}

int Calc::AddRandomNumber(int a) {
	return a + rng_->Get();
}
\end{lstlisting}

构造函数中,指针赋值给类字段。使用AddRandomNumber()中的这个字段来获取生成的值,生产代码将使用实数生成器,这些测试将使用模拟,需要解引用指针来启用多态性,也可以为不同的实现创建不同的生成器类。这里只需要一个:具有均匀分布的Mersenne Twister伪随机数生成器:

\begin{lstlisting}[style=styleCXX]
// chapter08/05-gmock/src/rng_mt19937.cpp

#include <random>
#include "rng_mt19937.h"
int RandomNumberGeneratorMt19937::Get() {
	std::random_device rd;
	std::mt19937 gen(rd());
	std::uniform_int_distribution<> distrib(1, 6);
	return distrib(gen);
}
\end{lstlisting}

这段代码的效率不是很高,但对于这个简单的示例来说已经足够了。其目的是生成从1到6的数字并将它们返回给调用者。这个类的头文件非常简单:

\begin{lstlisting}[style=styleCXX]
// chapter08/05-gmock/src/rng_mt19937.h

#include "rng.h"
class RandomNumberGeneratorMt19937
		: public RandomNumberGenerator {
public:
	int Get() override;
};
\end{lstlisting}

生产代码中使用的方式:

\begin{lstlisting}[style=styleCXX]
// chapter08/05-gmock/src/run.cpp

#include <iostream>
#include "calc.h"
#include "rng_mt19937.h"

using namespace std;
int run() {
	auto rng = new RandomNumberGeneratorMt19937();
	Calc c(rng);
	cout << "Random dice throw + 1 = "
		 << c.AddRandomNumber(1) << endl;
	delete rng;
	return 0;
}
\end{lstlisting}

这里已经创建了一个生成器,并将指向它的指针传递给Calc的构造函数。一切准备就绪,可以开始编写mock了。为了使事情有条理,开发人员通常将模拟放在一个单独的test/mocks目录中。为了防止歧义,头文件名有一个\_mock后缀。下面是我们要执行的代码:

\begin{lstlisting}[style=styleCXX]
// chapter08/05-gmock/test/mocks/rng_mock.h

#pragma once
#include "gmock/gmock.h"

class RandomNumberGeneratorMock : public
RandomNumberGenerator {
public:
	MOCK_METHOD(int, Get, (), (override));
};
\end{lstlisting}

添加了gmock.h头文件之后,可以声明我们的mock。按计划,它是一个实现RandomNumberGenerator接口的类。不需要自己编写方法,而是需要使用GMock提供的MOCK\_METHOD宏,告诉框架应该模拟接口中的哪些方法。使用以下格式(注意括号):

\begin{lstlisting}[style=styleCXX]
MOCK_METHOD(<return type>, <method name>,
(<argument list>), (<keywords>))
\end{lstlisting}

我们准备在测试套件中使用模拟(为了简洁起见,省略了之前的测试用例):

\begin{lstlisting}[style=styleCXX]
// chapter08/05-gmock/test/calc_test.cpp

#include <gtest/gtest.h>
#include "calc.h"
#include "mocks/rng_mock.h"

using namespace ::testing;
class CalcTestSuite : public Test {
	protected:
	RandomNumberGeneratorMock rng_mock_;
	Calc sut_{&rng_mock_};

};
TEST_F(CalcTestSuite, AddRandomNumberAddsThree) {
	EXPECT_CALL(rng_mock_,
	Get()).Times(1).WillOnce(Return(3));
	EXPECT_EQ(4, sut_.AddRandomNumber(1));
}
\end{lstlisting}

分析一下这些更改:加了新的头文件,并在测试套件中为rng\_mock\_创建了一个新字段。接下来,模拟的地址传递给sut\_的构造函数,因为字段是按声明顺序初始化的(rng\_mock\_在sut\_之前)。

测试用例中,在rng\_mock\_的Get()方法上调用GMock的EXPECT\_CALL宏,若在执行期间没有调用Get()方法,则测试失败。Times链式调用显式地声明了,通过测试必须发生多少个调用。WillOnce确定在调用方法后模拟框架做什么(它返回3)。

通过使用GMock,就能够在预期结果的同时表达模拟的行为。这极大地提高了可读性并简化了测试的维护。最重要的是,它在每个测试用例中都提供了足够的弹性,可以区分单个表达语句所发生的情况。

最后,需要确保gmock库与测试运行器链接。为了实现这一点,将它添加到target\_link\_libraries()中:

\begin{lstlisting}[style=styleCMake]
# chapter08/05-gmock/test/CMakeLists.txt

include(FetchContent)
FetchContent_Declare(
	googletest
	GIT_REPOSITORY https://github.com/google/googletest.git
	GIT_TAG release-1.11.0
)
# For Windows: Prevent overriding the parent project's
  compiler/linker settings
set(gtest_force_shared_crt ON CACHE BOOL "" FORCE)
FetchContent_MakeAvailable(googletest)
add_executable(unit_tests
				calc_test.cpp
				run_test.cpp)
target_link_libraries(unit_tests PRIVATE sut gtest_main
gmock)
include(GoogleTest)
gtest_discover_tests(unit_tests)
\end{lstlisting}

现在,可以享受GTest框架了。GTest和GMock都是非常高级的工具,具有用于不同场合的概念、程序和辅助程序。这个例子(尽管有点长)只触及了表面,但我鼓励您将它们合并到项目中,因为这将极大地提高项目的代码质量。从官方文档中的mock for Dummies页面开始学习GMock的一个好地方(可以在扩展阅读部分中找到该页面的链接)。

有了合适的测试,应该以某种方式衡量哪些测试了,哪些没有,并努力改善情况。最好使用自动工具来收集和报告这些信息。

