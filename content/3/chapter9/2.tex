专业开发人员通常遵循规则,说高级开发人员知道什么时候打破规则(他们可以证明需要这样做)。另一方面,据说非常资深的开发人员不会违反规则,因为不断向他人解释自己的理由是浪费时间。我说,选择你认可的方式即可,专注于真正重要的事情,对产品有实际影响的事情。

当谈到编码风格和格式时,程序员面临着无数的选择:我们应该使用制表符。还是空格来缩进?若是空格,有多少?一列的字符限制是多少?在文件中呢?大多数情况下,这样的选择不会影响程序的行为,但它们确实会产生很多杂声,并开始冗长的讨论,而这些讨论不会给产品带来太多价值。

有些做法是公认的,但大多数时候是在争论个人偏好。毕竟,在超过120的列中强制执行80个字符是一种随意的选择。选择什么并不重要,只要一致。风格一致是挺好,因为它会影响软件代码的可读性。

避免这种情况的最好方法是使用格式化工具,如clang-format。若代码的格式不正确我们会得到提醒,甚至可以进行修复。

下面是一个格式化代码的命令示例:

\begin{tcblisting}{commandshell={}}
clang-format -i --style=LLVM filename1.cpp filename2.cpp
\end{tcblisting}

-i选项告诉ClangFormat就地编辑文件。-{}-style选择应该使用的支持的格式化样式:LLVM,Google,Chromium, Mozilla, WebKit,或自定义(以文件方式提供)(在扩展阅读部分有详细的链接)。

当然,我们不希望每次进行更改时都手动执行此命令,CMake应该将此作为构建过程的一部分来处理。我们已经知道如何在系统中找到clang-format(需要手动安装)。

我们还没有讨论如何将外部工具应用到所有源文件的过程。为此,我们会创建一个方便的函数,可以放在cmake目录中:

\begin{lstlisting}[style=styleCMake]
# chapter09/01-formatting/cmake/Format.cmake

function(Format target directory)
	find_program(CLANG-FORMAT_PATH clang-format REQUIRED)
	set(EXPRESSION h hpp hh c cc cxx cpp)
	list(TRANSFORM EXPRESSION PREPEND "${directory}/*.")
	file(GLOB_RECURSE SOURCE_FILES FOLLOW_SYMLINKS
		LIST_DIRECTORIES false ${EXPRESSION}
	)
	add_custom_command(TARGET ${target} PRE_BUILD COMMAND
		${CLANG-FORMAT_PATH} -i --style=file ${SOURCE_FILES}
	)
endfunction()
\end{lstlisting}

Format函数接受两个参数:target和directory,在目标构建之前,其将格式化目录中的所有源文件。

从技术上讲,并非目录中的所有文件都必须属于目标(而且目标源可能位于多个目录中)。然而,查找属于目标(以及可能的依赖目标)的所有源文件和头文件是一个复杂的过程,特别是当需要过滤掉属于外部库,且是不应该格式化的头文件。这种情况下,处理目录更容易管理。我们可以为格式化目标的每个目录调用函数。

这个函数有以下步骤:

\begin{enumerate}
\item 
找到系统中安装的clang-format二进制文件。若没有找到二进制文件,REQUIRED关键字将停止配置并报错。

\item 
创建要格式化的文件扩展名列表(用作globbing表达式)。

\item 
每个表达式前面加上到目录的路径。

\item 
递归地搜索源和头文件(使用前面创建的列表),跳过目录,并将它们的路径放入SOURCE\_FILES变量中。

\item 
将格式化命令作为目标的PRE\_BUILD步骤。
\end{enumerate}

该命令将很好地用于中小型代码基础。对于大量的文件,需要将绝对文件路径转换为相对路径,并使用目录作为工作目录执行格式化(list(transform)在这里很有用)。这可能是必要的,因为传递给shell的命令有长度限制(通常约为13000个字符),太长的路径根本不适合。

来看看如何在实践中使用这个函数,将使用以下项目结构:

\begin{tcblisting}{commandshell={}}
- CMakeLists.txt
- .clang-format
- cmake
    |- Format.cmake
- src
    |- CMakeLists.txt
    |- header.h
    |- main.cpp
\end{tcblisting}

首先,需要设置项目,并将cmake目录添加到模块路径中,以便稍后包含:

\begin{lstlisting}[style=styleCMake]
# chapter09/01-formatting/CMakeLists.txt

cmake_minimum_required(VERSION 3.20.0)
project(Formatting CXX)
enable_testing()
list(APPEND CMAKE_MODULE_PATH "${CMAKE_SOURCE_DIR}/cmake")
add_subdirectory(src bin)
\end{lstlisting}

设置好之后,为src目录完成列表文件:

\begin{lstlisting}[style=styleCMake]
# chapter09/01-formatting/src/CMakeLists.txt

add_executable(main main.cpp)
include(Format)
Format(main .)
\end{lstlisting}

我们已经创建了一个可执行的目标主文件,包括Format.cmake模块,并为当前目录(src)中的主目标使用Format()函数。

现在,需要一些未格式化的源文件。头文件中只是一个简单的未使用的函数:

\begin{lstlisting}[style=styleCXX]
// chapter09/01-formatting/src/header.h

int unused() { return 2 + 2; }
\end{lstlisting}

我们还将添加一个有太多空格的源文件:

\begin{lstlisting}[style=styleCXX]
// chapter09/01-formatting/src/main.cpp

#include <iostream>
	using namespace std;
		int main() {
			
			cout << "Hello, world!" << endl;
		}
\end{lstlisting}

差不多准备好了。剩下的就是格式化器的配置文件(通过命令行中的-{}style=file参数启用):

\begin{lstlisting}[style=stylePython]
# chapter09/01-formatting/.clang-format

BasedOnStyle: Google
ColumnLimit: 140
UseTab: Never
AllowShortLoopsOnASingleLine: false
AllowShortFunctionsOnASingleLine: false
AllowShortIfStatementsOnASingleLine: false
\end{lstlisting}

Clang Format将扫描父目录以找到.clang-format文件,该文件指定了确切的格式化规则。这允许指定每个小细节,或者定制前面提到的标准之一。在这个例子中,选择从Google的编码风格开始,并进行了一些调整:将列限制为140个字符,删除制表符,并允许短循环、函数和if语句。

来看看在构建这个项目之后文件是如何变化的(在编译之前会自动格式化):

\begin{lstlisting}[style=styleCXX]
// chapter09/01-formatting/src/header.h (formatted)

int unused() {
	return 2 + 2;
}
\end{lstlisting}

头文件格式化了,即使它没有被目标使用,短函数不允许在一行中出现。正如预期的那样,格式化程序添加了新行。main.cpp文件现在看起来也很舒服:

\begin{lstlisting}[style=styleCXX]
// chapter09/01-formatting/src/main.cpp (formatted)

#include <iostream>
using namespace std;
int main() {
	cout << "Hello, world!" << endl;
}
\end{lstlisting}

删除不必要的空白,标准化缩进。

添加自动格式化器并不需要花费太多精力,而且会为代码检查方面节省大量时间。若曾经不得不修改提交以纠正一些空白,就会了解这种感觉,一致的格式会使代码更简洁。

\begin{tcolorbox}[colback=blue!5!white,colframe=blue!75!black,title=Note]
将格式化应用于现有的代码库,很可能会对存储库中的大多数文件引入一个大的一次性更改。若(或您的团队成员)有一些正在进行的工作,这可能会导致许多合并冲突。最好是在完成所有挂起的修改之后协调这些工作。若不行,则考虑逐步进行,也许是在每个目录的基础上。您的开发同事会感谢您的。
\end{tcolorbox}

格式化器是一个伟大而简单的工具,可以将代码的可视化方面整合在一起,但不是一个成熟的程序分析工具(主要关注空白)。要处理更高级的场景,我们更需要使用能够理解程序源代码的实用程序,来进行代码静态分析。