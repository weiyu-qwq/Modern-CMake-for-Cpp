即使对于非常有经验的开发人员来说,生成高质量的代码也不是一件容易的事。通过向解决方案添加测试,降低了在代码中犯错误的风险,但这还不足以避免更复杂的问题。每一个软件都包含如此多的细节,以至于跟踪它们会成为了一项复杂的工作,甚至会和维护产品的团队达成几十种约定和多种特殊的设计实践。

有些问题与编码风格有关:应该在代码中使用80列,还是120列?是否能使用std::bind或Lambda函数?是否可以使用C风格的数组?小函数应该在单行中定义吗?应该始终坚持使用auto,还是只在增加可读性时才使用?

理想情况下,还可以避免一般情况下不正确的语句:无限循环、使用标准库保留的标识符、无意的精度损失、冗余的if语句,以及其他“非最佳实践”(参见扩展阅读部分的参考资料)。

另一件事情是代码的现代化:随着C++的发展,提供了新的特性。跟踪并重构代码升级到最新标准也很困难。此外,手动工作会耗费太多时间,并可能引入错误,这对于大型代码库来说风险相当大。

最后,应该检查启动时是如何工作的:执行程序并检查其内存。使用后是否正确释放内存?是否正确地访问了初始化的数据?或者代码是否会解引用悬空指针?

手动管理所有这些问题低效且容易出错,可以使用程序来检查和执行规则,修复错误,并更新代码。是时候使用工具分析程序,代码将在每次构建时进行检查,以确保它符合标准。

本章中,我们将讨论以下主题:

\begin{itemize}
\item 
格式化

\item 
静态检查

\item 
Valgrind的动态分析
\end{itemize}













