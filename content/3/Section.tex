

Completing the previous sections has turned you into a self-sufficient build engineer capable of building all kinds of projects with CMake. The final step in becoming a CMake professional is learning how to introduce and automate various quality checks and prepare your projects for collaborative work and publication. High-quality projects developed within large companies often share the same philosophy: automate repetitive tasks that drain mental energy from important decisions.

To achieve that, we're leveraging the power of the CMake ecosystem to add all kinds of tests done during the build: code-style checks, unit tests, and static and dynamic analyses of our solutions. We'll also simplify the documentation process by using tooling to generate pretty web pages, and we'll package and install our project to make its consumption a breeze, both for other developers and end users.

As a summary, we will put together everything we have learned into one coherent
unit: a professional project that will stand the test of time.

\begin{itemize}
\item Chapter 8, Testing Frameworks
\item Chapter 9, Program Analysis Tools
\item Chapter 10, Generating Documentation
\item Chapter 11, Installing and Packaging
\item Chapter 12, Creating Your Professional Project
\end{itemize}